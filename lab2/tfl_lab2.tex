\documentclass[12pt]{article}

\usepackage[utf8]{inputenc}
\usepackage[T2A]{fontenc}
\usepackage[russian]{babel}

\usepackage{amsmath,amssymb}
\usepackage{graphicx}
\usepackage{booktabs}
\usepackage{geometry}
\usepackage{listings}
\usepackage{xcolor}
\usepackage{float}

\geometry{margin=2cm}

\lstset{
  basicstyle=\ttfamily\small,
  columns=fullflexible,
  breaklines=true,
  frame=single
}

\title{Лабораторная работа 2\\(вариант 1)}
\author{}
\date{}

\begin{document}
\maketitle

\section*{1. Исходное регулярное выражение}

Алфавит: $\Sigma=\{a,b\}$.

Регулярное выражение:
\[
R = \big((a|b)^\ast\,abb\,(a|b)\big)\;|\;\big((a|ba)^\ast\,bb\,(a^\ast b)^\ast\,aba\,(a|b)\big).
\]

\section*{2. НКА}

Построенный НКА изображён на рис.~\ref{fig:nfa}.

\begin{figure}[H]
    \centering
    \includegraphics[width=0.95\linewidth]{NFA.png}
    \caption{НКА}
    \label{fig:nfa}
\end{figure}

\subsection*{2.1. Таблица префиксов и суффиксов НКА}

\begin{table}[H]
\centering
\small
\setlength{\tabcolsep}{6pt}
\renewcommand{\arraystretch}{1.2}
\begin{tabular}{lccccccc}
\toprule
 & ba & $\varepsilon$ & bba & b & abab & babaa & abba \\
\midrule
bab   & 1 & 0 & 0 & 0 & 0 & 1 & 1 \\
abba  & 0 & 1 & 1 & 0 & 0 & 1 & 1 \\
a     & 0 & 0 & 1 & 0 & 0 & 0 & 1 \\
abb   & 0 & 0 & 0 & 1 & 1 & 1 & 1 \\
bb    & 0 & 0 & 0 & 0 & 1 & 1 & 1 \\
b     & 0 & 0 & 0 & 0 & 0 & 1 & 1 \\
$\varepsilon$ & 0 & 0 & 0 & 0 & 0 & 0 & 1 \\
\bottomrule
\end{tabular}
\caption{Таблица префиксов и суффиксов НКА}
\end{table}



\section*{3. ДКА и минимизация}

На рис.~\ref{fig:dfa} показан минимальный ДКА.

\begin{figure}[H]
\centering
\includegraphics[width=0.98\linewidth]{DFA.png}
\caption{минимальный ДКА}
\label{fig:dfa}
\end{figure}

\subsection*{3.1. Классы эквивалентности}

Отметим классы эквивалентности: рис.~\ref{fig:eq}.

\begin{figure}[H]
\centering
\includegraphics[width=0.98\linewidth]{EQ.png}
\caption{Минимальный ДКА}
\label{fig:eq}
\end{figure}

\subsection*{3.2. Проверка минимальности}

Построим таблицу эквивалентности.
Если все строки различны, то состояния попарно различны, значит ДКА минимален.

\begin{table}[H]
\centering
\tiny
\setlength{\tabcolsep}{3pt}
\renewcommand{\arraystretch}{1.15}
\begin{tabular}{lccccccccc}
\toprule
 & $\varepsilon$ & a & aa & abaa & abba & ba & baa & babaa & bba \\
\midrule
$\varepsilon$ & 0 & 0 & 0 & 0 & 1 & 0 & 0 & 0 & 0 \\
a            & 0 & 0 & 0 & 0 & 1 & 0 & 0 & 0 & 1 \\
b            & 0 & 0 & 0 & 0 & 1 & 0 & 0 & 1 & 0 \\
ab           & 0 & 0 & 0 & 0 & 0 & 1 & 0 & 1 & 0 \\
abb          & 0 & 1 & 0 & 1 & 1 & 0 & 0 & 1 & 0 \\
abba         & 1 & 0 & 0 & 0 & 0 & 0 & 1 & 1 & 1 \\
abbb         & 1 & 0 & 0 & 1 & 1 & 0 & 0 & 1 & 0 \\
bb           & 0 & 0 & 0 & 1 & 1 & 0 & 0 & 1 & 0 \\
bba          & 0 & 0 & 0 & 0 & 1 & 0 & 0 & 1 & 1 \\
bbaa         & 0 & 0 & 0 & 0 & 1 & 0 & 1 & 1 & 1 \\
bbaab        & 0 & 0 & 0 & 1 & 1 & 1 & 0 & 1 & 0 \\
bbab         & 0 & 0 & 1 & 1 & 1 & 1 & 0 & 1 & 0 \\
bbaba        & 0 & 1 & 0 & 0 & 1 & 0 & 1 & 1 & 1 \\
bbabaa       & 1 & 0 & 0 & 0 & 1 & 0 & 0 & 1 & 1 \\
bbabab       & 1 & 0 & 1 & 1 & 1 & 1 & 1 & 1 & 0 \\
\bottomrule
\end{tabular}
\caption{Таблица различимости}
\end{table}

Все строки различны, значит он минимален.

\section*{4. Расширенное регулярное выражение и ПКА}

\subsection*{4.1. Расширенная регулярка}

Исходное выражение:
\[
R = \big((a|b)^\ast\,abb\,(a|b)\big)\;\;|\;\;\big((a|ba)^\ast\,bb\,(a^\ast b)^\ast\,aba\,(a|b)\big).
\]

Перепишем его в расширенном виде:
\[
R \;=\; \texttt{\^{}(.*((?<=abb)|(?=bb).*(?<=baba)).)\$}
\]

Внешняя структура имеет вид
\[
.* \; \alpha \; .
\]
Первая часть \texttt{.*} задаёт произвольный префикс слова.
Последний символ \texttt{.} соответствует одному произвольному символу алфавита.

Внутреннее выражение содержит две альтернативы.

\medskip
\textbf{Первая альтернатива:} \texttt{(?<=abb)}.
Она требует, чтобы непосредственно перед текущей позицией
стояло подслово \texttt{abb}.
Поскольку далее следует последний символ,
всё слово оканчивается на \texttt{abbx},
где $x \in \{a,b\}$.
То есть, его суффикс равен \texttt{abba} или \texttt{abbb}.

\medskip
\textbf{Вторая альтернатива:} \texttt{(?=bb).*(?<=baba)}.

Конструкция \texttt{(?=bb)} требует, чтобы начиная с текущей позиции
в слове стояло подслово \texttt{bb}.
Тем самым фиксируется существование некоторого вхождения \texttt{bb}.

После этого допускается произвольная последовательность символов \texttt{.*},
а затем \texttt{(?<=baba)} требует, чтобы непосредственно перед последним символом стояло подслово \texttt{baba}.

Следовательно, во второй альтернативе слово
обязательно содержит подслово \texttt{bb}
и оканчивается на \texttt{babax}, где $x \in \{a,b\}$.

\medskip
Таким образом, язык состоит из всех слов над $\{a,b\}$,
которые либо имеют суффикс \texttt{abb(a|b)},
либо содержат подслово \texttt{bb}
и одновременно имеют суффикс \texttt{baba(a|b)}.

\subsection*{4.2. ПКА}

ПКА для того же языка изображён на рис.~\ref{fig:afa}.

\begin{figure}[H]
\centering
\includegraphics[width=0.95\linewidth]{AFA.png}
\caption{ПКА}
\label{fig:afa}
\end{figure}

\begin{table}[H]
\centering
\small
\setlength{\tabcolsep}{6pt}
\renewcommand{\arraystretch}{1.2}
\begin{tabular}{lcccccc}
\toprule
 & $\varepsilon$ & a & abaa & ba & baa & babaa \\
\midrule
bb   & 0 & 1 & 1 & 1 & 1 & 1 \\
bab  & 0 & 0 & 1 & 1 & 1 & 1 \\
b    & 0 & 0 & 0 & 1 & 1 & 1 \\
baba & 0 & 0 & 0 & 0 & 1 & 1 \\
ba   & 0 & 0 & 0 & 0 & 0 & 1 \\
a    & 0 & 0 & 0 & 0 & 0 & 0 \\
\bottomrule
\end{tabular}
\caption{Таблица префиксов и суффиксов для ПКА}
\end{table}



\end{document}
