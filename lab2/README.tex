\documentclass[12pt]{article}

\usepackage[utf8]{inputenc}
\usepackage[T2A]{fontenc}
\usepackage[russian]{babel}

\usepackage{amsmath,amssymb}
\usepackage{graphicx}
\usepackage{booktabs}
\usepackage{geometry}
\usepackage{listings}
\usepackage{xcolor}
\usepackage{float}

\geometry{margin=2cm}

\lstset{
  basicstyle=\ttfamily\small,
  columns=fullflexible,
  breaklines=true,
  frame=single
}

\title{Лабораторная работа 2\\(вариант 1)}
\author{}
\date{}

\begin{document}
\maketitle

\section*{1. Исходное регулярное выражение}

Алфавит: $\Sigma=\{a,b\}$.

Регулярное выражение:
\[
R = \big((a|b)^\ast\,abb\,(a|b)\big)\;|\;\big((a|ba)^\ast\,bb\,(a^\ast b)^\ast\,aba\,(a|b)\big).
\]

\section*{2. НКА}

Построенный НКА изображён на рис.~\ref{fig:nfa}.

\begin{figure}[H]
    \centering
    \includegraphics[width=0.95\linewidth]{NFA.png}
    \caption{НКА}
    \label{fig:nfa}
\end{figure}

\subsection*{2.1. Таблица префиксов и суффиксов НКА}

\begin{table}[H]
\centering
\scriptsize
\setlength{\tabcolsep}{4pt}
\renewcommand{\arraystretch}{1.15}
\begin{tabular}{lcccccccccc}
\toprule
 & $\varepsilon$ & a & aa & abab & ababaa & abba & ba & baa & babaa & bba \\
\midrule
$\varepsilon$ & 0 & 0 & 0 & 0 & 0 & 1 & 0 & 0 & 0 & 0 \\
a            & 0 & 0 & 0 & 0 & 0 & 1 & 0 & 0 & 0 & 1 \\
ab           & 0 & 0 & 0 & 0 & 0 & 1 & 1 & 0 & 1 & 0 \\
abb          & 0 & 1 & 0 & 1 & 1 & 1 & 0 & 0 & 1 & 0 \\
abba         & 1 & 0 & 0 & 0 & 1 & 1 & 0 & 1 & 1 & 1 \\
b            & 0 & 0 & 0 & 0 & 0 & 1 & 0 & 0 & 1 & 0 \\
bb           & 0 & 0 & 0 & 1 & 1 & 1 & 0 & 0 & 1 & 0 \\
bba          & 0 & 0 & 0 & 0 & 1 & 1 & 0 & 1 & 1 & 1 \\
bbaa         & 0 & 0 & 0 & 0 & 1 & 1 & 0 & 0 & 1 & 1 \\
bbab         & 0 & 0 & 1 & 1 & 1 & 1 & 1 & 0 & 1 & 1 \\
\bottomrule
\end{tabular}
\caption{Таблица префиксов и суффиксов для НКА}
\end{table}

\section*{3. ДКА и минимизация}

На рис.~\ref{fig:dfa} показан минимальный ДКА.

\begin{figure}[H]
\centering
\includegraphics[width=0.98\linewidth]{DFA.png}
\caption{минимальный ДКА}
\label{fig:dfa}
\end{figure}

\subsection*{3.1. Классы эквивалентности}

Отметим классы эквивалентности: рис.~\ref{fig:eq}.

\begin{figure}[H]
\centering
\includegraphics[width=0.98\linewidth]{EQ.png}
\caption{Минимальный ДКА}
\label{fig:eq}
\end{figure}

\subsection*{3.2. Проверка минимальности}

Построим таблицу эквивалентности.
Если все строки различны, то состояния попарно различны, значит ДКА минимален.

\begin{table}[H]
\centering
\tiny
\setlength{\tabcolsep}{3pt}
\renewcommand{\arraystretch}{1.15}
\begin{tabular}{lccccccccc}
\toprule
 & $\varepsilon$ & a & aa & abaa & abba & ba & baa & babaa & bba \\
\midrule
$\varepsilon$ & 0 & 0 & 0 & 0 & 1 & 0 & 0 & 0 & 0 \\
a            & 0 & 0 & 0 & 0 & 1 & 0 & 0 & 0 & 1 \\
b            & 0 & 0 & 0 & 0 & 1 & 0 & 0 & 1 & 0 \\
ab           & 0 & 0 & 0 & 0 & 0 & 1 & 0 & 1 & 0 \\
abb          & 0 & 1 & 0 & 1 & 1 & 0 & 0 & 1 & 0 \\
abba         & 1 & 0 & 0 & 0 & 0 & 0 & 1 & 1 & 1 \\
abbb         & 1 & 0 & 0 & 1 & 1 & 0 & 0 & 1 & 0 \\
bb           & 0 & 0 & 0 & 1 & 1 & 0 & 0 & 1 & 0 \\
bba          & 0 & 0 & 0 & 0 & 1 & 0 & 0 & 1 & 1 \\
bbaa         & 0 & 0 & 0 & 0 & 1 & 0 & 1 & 1 & 1 \\
bbaab        & 0 & 0 & 0 & 1 & 1 & 1 & 0 & 1 & 0 \\
bbab         & 0 & 0 & 1 & 1 & 1 & 1 & 0 & 1 & 0 \\
bbaba        & 0 & 1 & 0 & 0 & 1 & 0 & 1 & 1 & 1 \\
bbabaa       & 1 & 0 & 0 & 0 & 1 & 0 & 0 & 1 & 1 \\
bbabab       & 1 & 0 & 1 & 1 & 1 & 1 & 1 & 1 & 0 \\
\bottomrule
\end{tabular}
\caption{Таблица различимости}
\end{table}

Все строки различны, значит он минимален.

\section*{4. Расширенное регулярное выражение и ПКА}

\subsection*{4.1. Расширенная регулярка}

Исходное выражение:
\[
R = \big((a|b)^\ast\,abb\,(a|b)\big)\;\;|\;\;\big((a|ba)^\ast\,bb\,(a^\ast b)^\ast\,aba\,(a|b)\big).
\]

Перепишем его в расширенном виде:
\[
R \;=\; \hat{}\Big(
  (a|b)^\ast\, (?<=abb)\,(a|b)
  \;\;|\;\;
  (a|b)^\ast\, (?=bb)\,(a|b)^\ast\, (?<=baba)\,(a|b)
\Big)\$.
\]

\paragraph{Почему первая ветка эквивалентна $(a|b)^\ast abb(a|b)$.}
Конструкция $(a|b)^\ast$ разрешает произвольный префикс над $\{a,b\}$.
Далее $abb(a|b)$ означает, что слово заканчивается на $abba$ или на $abbb$.

Это записано как $(a|b)^\ast (?<=abb)(a|b)$:
проверка $(?<=abb)$ требует, чтобы прямо перед последним символом стояло $abb$,
а затем $(a|b)$ задаёт этот последний символ.

\paragraph{Почему вторая ветка эквивалентна $(a|ba)^\ast\,bb\,(a^\ast b)^\ast\,aba\,(a|b)$}.
Заметим, что как только в слове появляется подслово \texttt{bb}, всё, что стоит до этого подслова,
перестаёт иметь значение для распознавания:
дальше достаточно проверить, что \texttt{bb} действительно встречается где-то внутри слова,
и что ближе к концу выполняется все остальное.

Поэтому префикс до некоторого вхождения \texttt{bb} можно обозначить просто как $(a|b)^\ast$,
а сам факт существования позиции, с которой начинается \texttt{bb}, зафиксируем $(?=bb)$.

Часть $(a^\ast b)^\ast$ порождает строки, которые либо пусты, либо оканчиваются на $b$
(потому что каждый блок $a^\ast b$ заканчивается символом $b$).
Значит, перед подсловом \texttt{aba} в слове может стоять либо ничего,
либо $b$.

Отсюда следует, что перед последним символом слова должно стоять подслово \texttt{baba}. $(a|b)$ задаёт последний символ слова.

Это записано как:
\[
(a|b)^\ast (?=bb)\,(a|b)^\ast (?<=baba)\,(a|b),
\]
где $(?<=baba)$ гарантирует наличие \texttt{baba} прямо перед последним символом.

\subsection*{4.2. ПКА}

ПКА для того же языка изображён на рис.~\ref{fig:afa}.

\begin{figure}[H]
\centering
\includegraphics[width=0.95\linewidth]{AFA.png}
\caption{ПКА}
\label{fig:afa}
\end{figure}

\begin{table}[H]
\centering
\scriptsize
\setlength{\tabcolsep}{5pt}
\renewcommand{\arraystretch}{1.15}
\begin{tabular}{lcccccc}
\toprule
 & $\varepsilon$ & a & ababaa & baa & babaa & bba \\
\midrule
a     & 0 & 0 & 0 & 0 & 0 & 1 \\
ab    & 0 & 0 & 0 & 0 & 1 & 0 \\
abba  & 1 & 0 & 1 & 1 & 1 & 1 \\
bb    & 0 & 0 & 1 & 0 & 1 & 0 \\
bba   & 0 & 0 & 1 & 1 & 1 & 1 \\
bbab  & 0 & 0 & 1 & 0 & 1 & 1 \\
\bottomrule
\end{tabular}
\caption{Таблица префиксов и суффиксов для ПКА}
\end{table}

\end{document}