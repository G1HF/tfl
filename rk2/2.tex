\documentclass[12pt]{article}

\usepackage[T2A]{fontenc}
\usepackage[utf8]{inputenc}
\usepackage[russian]{babel}
\usepackage{amsmath}

\begin{document}

\section*{Рк 2. Задача 2. Решение}

Рассмотрим язык
\[
L=\left\{\, w_1 a w_2 w_3 \;\middle|\; |w_1|>0 \;\land\;
\bigl( w_1=w_2^R \;\lor\; w_1=w_3^R \bigr)\right\}.
\]

Класс DCFL замкнут относительно дополнения и пересечения с регулярными языками.
Значит, если $L$ --- DCFL, то язык
\[
L'=\overline{L}\cap R
\]
тоже DCFL, а следовательно, КС.

Возьмем регулярный язык
\[
R = c b^{+} c \; a \; c b^{+} c \; c \; b^{+} c.
\]

Достаточно показать, что $L'$ не КС.
\newline
\newline
Пусть $p$ --- константа из леммы Огдена. Обозначим
\[
M=\text{НОК}(1,2,\dots,p)
\]

Рассмотрим слово
\[
w = c b^{p} c \; a \; c b^{p+M} c \; c \; b^{p+M} c.
\]
Очевидно, $w\in R$.

Покажем, что $w\notin L$. Здесь
\[
w_1 = c b^{p} c.
\]
Это палиндром, поэтому $w_1^R = w_1$. Тогда условия
\[
w_1=w_2^R \quad \text{или} \quad w_1=w_3^R
\]
эквивалентны соответственно
\[
w_2 = c b^{p} c \quad \text{или} \quad w_3 = c b^{p} c.
\]

Хвост после символа $a$ равен
\[
c b^{p+M} c \; c \; b^{p+M} c.
\]
Ни один его префикс не равен $c b^{p} c$, поскольку после первой буквы $c$ идут $p+M$ букв $b$ до следующей $c$, а не $p$.
По той же причине ни один суффикс хвоста не равен $c b^{p} c$.
Значит, не существует разбиения хвоста на $w_2w_3$, при котором выполнялось бы $w_2=cb^{p}c$ или $w_3=cb^{p}c$.

Следовательно, $w\notin L$, то есть $w\in \overline{L}$, и значит $w\in L'=\overline{L}\cap R$.
\newline
\newline
Отметим $p$ позиций внутри блока $b^{p}$ в слове $w$ (в части $w_1$).

Пусть для слова $w$ выполнено разбиение из леммы Огдена:
\[
w=uvxyz,
\]
причём $v$ и $y$ содержат хотя бы одну отмеченную позицию.

Если $v$ или $y$ задевает символ $c$, то при накачке с $i=0$ нарушается форма префикса $cb^{+}c$, и полученное слово не лежит в $R$.
Значит, оно не лежит в $L'$.

Иначе $v$ и $y$ состоят только из букв $b$ внутри отмеченного блока $b^{p}$.
Пусть $|v|=x$, $|y|=y$. Тогда $x\ge 1$, $y\ge 1$ и $x+y\le p$.

Выберем
\[
i=\frac{M}{x+y}+1.
\]
Это целое число, поскольку $M$ делится на любое число от $1$ до $p$, в частности на $x+y$.

Тогда длина первого блока $b$ после накачки станет
\[
p+(i-1)(x+y)=p+M,
\]
то есть накачанное слово имеет вид
\[
c b^{p+M} c \; a \; c b^{p+M} c \; c \; b^{p+M} c,
\]
и, следовательно, принадлежит $R$.

Теперь покажем, что это слово принадлежит $L$.
Выберем разбиение хвоста после $a$ так, чтобы
\[
w_2 = c b^{p+M} c,
\]
то есть $w_2$ --- это первый блок после $a$.
Так как $c b^{p+M} c$ является палиндромом, имеем $w_2^R=w_2$, и потому
\[
w_1 = c b^{p+M} c = w_2^R.
\]
Значит, накачанное слово лежит в $L$, а значит не лежит в $\overline{L}$, следовательно не лежит в $L'$.

Получили противоречие с леммой Огдена. Следовательно, $L'$ не КС.

Значит, $L'$ не DCFL. Если бы $L$ был DCFL, то $L'$ был бы DCFL.
Противоречие. Значит, $L$ не DCFL.

\end{document}