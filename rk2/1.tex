\documentclass[12pt]{article}

\usepackage[T2A]{fontenc}
\usepackage[utf8]{inputenc}
\usepackage[russian]{babel}
\usepackage{amsmath}

\begin{document}

\section*{Рк 2. Задача 1. Решение}

\subsection*{Доказательство контекстной свободы языка}

Чтобы определить, является ли слово палиндромом, нужно
убирать символы с боков. Если в какой-то момент символы окажутся разными,
то слово не является палиндромом.

Заметим, что правило
\[
S \rightarrow aSa
\]
обычным образом проверяет совпадение символов.

В правиле
\[
S \rightarrow bSSb
\]
можно дополнительно проверить два символа: один в начале первой \(S\), а другой в конце второй.
Тогда можно ввести два правила, которые делают слово непалиндромом сразу после первого применения:
\[
S \rightarrow baSa\,bSSbb, \qquad
S \rightarrow bbSS\,baSab.
\]

Таким образом, будут убраны два различных символа с боков.
Значит, необходимо и достаточно потребовать применение
хотя бы одного из таких правил.

В результате получаем грамматику:
\[
\begin{aligned}
S &\rightarrow baTabTTbb \\
S &\rightarrow bbTTbaTab \\
S &\rightarrow bSSb \\
S &\rightarrow aSa \\
T &\rightarrow \varepsilon \\
T &\rightarrow aTa \\
T &\rightarrow bTTb
\end{aligned}
\]

Следовательно, язык является КС.

\subsection*{Доказательство нерегулярности языка}

Рассмотрим слово
\[
w = a^{n} baabbba^{n}.
\]

Применяя лемму о накачке для регулярных языков, накачиваем часть
из первых символов \(a\) и получаем слово
\[
w' = a^{n+k} baabbba^{n}.
\]

Это слово не принадлежит исходному языку, так как с правого края мы быстрее уберем буквы \(a\), чем с левого,  и баланс нарушится.

Следовательно, язык не является регулярным.

\end{document}
